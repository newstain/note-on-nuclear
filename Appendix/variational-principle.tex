\chapter{变分原理及其应用}

%%%%%%%%%%%%%%%%%%%%%%%%%%%%%%
\section{变分原理}
在二维直角坐标中有两个定点$(x_1,\ y_1)$、$(x_2,\ y_2)$,连接这两点的任意一条曲线为$y = y(x)$,其满足边界条件如下:
\begin{equation}
    y(x_1) = y_1, \quad y(x_2) = y_2
\end{equation} 
现在,我们定义一个新的函数$f$,它是关于$y$和$y^{\prime}$($y$的一阶导数)的函数,形式为$f = f(y, y^{\prime})$,我们对其做关于$x$的定积分,如下
\begin{equation}
	I = \int_{x_1}^{x_2} f(y, y^{\prime}) \, dx	\label{eq:variation-simple}
\end{equation} 
这样,当$y$变化时,定积分$I$也会随之改变,这样,我们期望能找到某一个$y$,使$I$有极值(极大或极小)。

我们回顾一下以往的极值问题,对于某一个函数$y=y(x)$,我们变化自变量$x$,通过对比不同$x$处的函数值$y$来找到极值;现在我们来看式\eqref{eq:variation-simple},它的积分变量是$x$,且积分区间固定,我们无法通过改变$x$来改变$I$的值,但是被积函数$f$是关于$y$和$y^{\prime}$的函数,它的形式并不确定,因此,针对式$\eqref{eq:variation-simple}$的变分问题,我们考察的是:取不同形式的$y(x)$,使积分$I$有极值。\CJKunderdot{\textcolor{red}{我们习惯上把$I$称作$y(x)$的泛函}}【也就是函数的函数,$f$是$y$的函数,$I$是$f$的函数;或者$y$是$x$的函数($x$变化范围,例如从$x=1$变为$x=2$),$I$是$y$的函数($y$变换形式,如$y=x$变为$y=x^2$)】。由于取的是$I$的极值,因此当我们对$y(x)$做微小的变化是,$I$在一级近似下应该不变(也就是$I$关于$y$的一阶导为0)。$\delta y$是$y$的无穷小变化,习惯上把\CJKunderdot{\textcolor{red}{$\delta y$称为$y$的变分}}。

当我们变化$\delta y$时,$f$的变化为
\begin{equation}
	\delta f = \frac{\partial f}{\partial y} \delta y + \frac{\partial f}{\partial y^{\prime}} \delta y^{\prime}
\end{equation} 
$I$相应的变化为
\begin{equation}
	\delta I = \int_{x_1}^{x_2} \delta f \, dx = \int_{x_1}^{x_2} \left[  \frac{\partial f}{\partial y} \delta y + \frac{\partial f}{\partial y^{\prime}} \delta y^{\prime} \right] \, dx	\label{eq:variation}
\end{equation} 
方括号第二项用分部积分方法
\begin{equation}
	\begin{aligned}
		\int_{x_1}^{x_2} \frac{\partial f}{\partial y^{\prime}} \delta y^{\prime}\, dx =&{} \int_{x_1}^{x_2} \frac{\partial f}{\partial y^{\prime}} \left( \frac{d (\delta y)}{dx} \right) \, dx = \int_{x_1}^{x_2} \frac{\partial f}{\partial y^{\prime}} \, d (\delta y)	\\
		=&{} \left. \left( \frac{\partial f}{\partial y^{\prime}} \delta y \right) \right|_{x_1}^{x_2} - \int_{x_1}^{x_2} \delta y \frac{d}{dx}\left( \frac{\partial f}{\partial y^{\prime}} \right) \, dx
	\end{aligned}
\end{equation} 
考虑到边界是固定的,也就是在$x=x_1$和$x=x_2$出函数值$y(x_1)$和$y(x_2)$保持不变,因此我们有$\delta y(x_1) = \delta y(x_2) = 0$,上式最后一个等号右边第一项为0,即
\begin{equation}
	\left[ \frac{\partial f}{\partial y^{\prime}} \delta y \right]_{x_1}^{x_2} = 0
\end{equation}
将以上两式带回到式\eqref{eq:variation},得到
\begin{equation}
	\delta I = \int_{x_1}^{x_2} \left[ \frac{\partial f}{\partial y} - \frac{d}{dx}\left( \frac{\partial f}{\partial y^{\prime}} \right) \right] \, dx
\end{equation} 
若要求$I$有极值,则$\delta I = 0$,那么有下式成立
\begin{equation}
    \frac{\partial f}{\partial y} - \frac{d}{dx}\left( \frac{\partial f}{\partial y^{\prime}} \right) = 0
\end{equation} 
上式则是\textcolor{red}{Euler-Lagrange方程}。

\section{变分原理与Schr\"{o}dinger方程}
我们可以从变分原理出发,推导出Schr\"{o}dinger方程。

体系的Hamilton量为$\hat{H}$,


