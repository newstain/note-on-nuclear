\chapter{刚体运动的动能}

\section{刚体的独立坐标}

刚体系统具有$N$个粒子,这些粒子间的相对位置是确定的,也就是说对于任意两个粒子$i$和$j$,其相对距离满足
\begin{equation}
	r_{ij} = c_{ij}	\label{eq:rigid-constraint}
\end{equation} 
其中$c_{ij}$是一个常数。这是$N$粒子刚体系统满足的约束方程。确定这样一个系统的空间状态,我们需要确定$N$个粒子的自由度,整个系统总共就有$3N$个自由度,这是非常复杂的,而上式所确定的约束方程总共有$\dfrac{N(N-1)}{2}$个,对于很大的$N$来说,这远远大于了系统的所有粒子的自由度数目,因此上述的约束方程并不是完全独立的。

在确定整个刚体系统的空间位置前,我们先考虑如何确定一个点在空间中的位置。对于空间中一个点$4$的位置,我们可以先确定空间中不同直线上的三个点$1$、$2$、$3$的位置,并且确切地知道点$4$相对于这三个点的距离时,我们就可以唯一确定点$4$的位置。现在考虑多粒子系统的刚体系统,我们可以知道任意一点$i$相对于$1$、$2$、$3$的距离$c_{i1}$、$c_{i2}$、$c_{i3}$,这样,我们基于这三个点就可以确定刚体内其余所有点的位置,也就确定了这个刚体的空间位置。

现在,我们已经把整个刚体的空间位置由$N$个点的自由度约化成了确定$3$个点的自由度问题,下面来定出这$3$个点所需要的自由度个数。首先,我们知道了刚体内所有点的约束条件为式\eqref{eq:rigid-constraint},从而知道这三个点具有确定的相对距离为
\begin{equation*}
	r_{12} = c_{12}, \quad r_{13} = c_{13}, \quad r_{23} = c_{23}
\end{equation*} 
我们确定点$1$需要三个自由度;点$2$在以点$1$为球心,$r_{12}$为半径的球面上,因此我们只需要知道点$2$的两个自由度便可以确定其位置;在确定前两个点的位置后,点$3$位于点$1$和点$2$为球心,$r_{13}$和$r_{23}$为半径的球面相交所构成的圆上,因此,我们只需要确定其中一个自由度就可以知道点$3$的确定位置。综上,我们总共需要确定$3+2+1=6$个自由度便可以最终定下三个点的具体空间位置,再根据刚体内点的约束关系,就可以确定好整个系统的空间位置。

\paragraph*{相对坐标系统:}空间中有一套Cartesian坐标系统,刚体处于这个坐标系统中,称为unprimed坐标系统,单位矢量为$\bf{i}$、$\bf{j}$和$\bf{k}$;在刚体内部建立另一套Cartesian坐标系统,刚体内部的粒子用此坐标系统来标记,称为primed坐标系统,其单位矢量为$\bf{i}^{\prime}$、$\bf{j}^{\prime}$和$\bf{k}^{\prime}$。

