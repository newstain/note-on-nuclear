\chapter{原子核平均场和多核子组态}

%%%%%%%%%%%%%%%%%%%%%%%%%%%%%%%%%%%%%%%%%
\section{原子核平均场}
\paragraph*{准粒子} 准粒子一般用来作为粒子近似。准粒子系统可以当成无相互作用的准粒子进行处理。
\paragraph*{平均场近似} 我们通常将强相互作用的粒子系统转换到弱相互作用的准粒子系统来处理多体问题。以下将讨论所谓的平均场(或Hartree-Fock)准粒子。在像原子核这样的多体系统中,核子-核子相互作用可忽略掉三体及以上的多体相互作用,最高写成两体相互作用的形式,其哈密顿量由动能项和势能项组成:
\begin{equation}
	H = T + V = \sum_{i = 1}^{A} t(\boldsymbol{r}_i) + \sum_{i, j = 1; i < j}^{A} v(\boldsymbol{r}_i, \boldsymbol{r}_j) =  \sum_{i = 1}^{A} \frac{-\hbar^2}{2m_N}\nabla_i^2 + \sum_{i, j = 1; i < j}^{A} v(\boldsymbol{r}_i, \boldsymbol{r}_j) 
\end{equation}
