\chapter{三维空间的量子力学问题}
\section{径向解}
中心势场问题中的径向方程写为如下形式:
\begin{equation}
	\frac{1}{R}\frac{d}{dr}\left(r^2 \frac{dR}{dr}\right) + \left\{\frac{2mr^2}{\hbar^2}[ E-V(r) ]\right\}=  l(l+1)	\label{radial-equation}
\end{equation} 
束缚态下的体系能量呈现分立谱,以下只考虑束缚态情况。

为求解方程\eqref{radial-equation},作函数代换以方便求解:
\begin{equation}
    R(r) = \frac{u(r)}{r}	\label{eq:radial-sub}
\end{equation} 
这样,我们可以得到如下关系:
\begin{equation}
	\begin{aligned}
		\frac{dR}{dr} =& \frac{d}{dr} \left( \frac{u(r)}{r} \right) = \frac{r (du/dr) - u}{r^2}	\\
		\frac{d}{dr}\left( r^2 \frac{dR}{dr} \right)  =& \frac{d}{dr}\left( r \frac{du}{dr} - u \right) = r \frac{d^2u}{dr^2} 
	\end{aligned}
	\label{eq:radila-sub-tran}
\end{equation} 
将式\eqref{eq:radial-sub}、\eqref{eq:radila-sub-tran}代入式\eqref{radial-equation},得到关于$u(r)$的方程,
\begin{equation}
	-\frac{\hbar^2}{2m} \frac{d^2 u}{dr^2} + \left[ V(r) + \frac{\hbar^2}{2m}\frac{l(l+1)}{r^2} \right] u = Eu
\end{equation} 
这个方程与一维薛定谔方程类似,不过势能项有所改变,其中的$\frac{\hbar^2}{2m}\frac{l(l+1)}{r^2}$类似与经典力学中的离心作用,称为离心项。

下面在谐振子势阱下对其进行求解。

\subsection{谐振子势下的径向方程求解}

径向上的球形谐振子势表示为:
\begin{equation*}
	V = \frac{1}{2} \omega^{2} r^{2}
\end{equation*} 


